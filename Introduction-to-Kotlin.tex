\documentclass[a4paper, 11pt]{article}
\usepackage[a4paper, left=3cm, right=3cm, top=3cm, bottom=3cm]{geometry}
\usepackage[dvipsnames]{xcolor}
\usepackage{listings}
\usepackage{titling}
\usepackage[T1]{fontenc}
\usepackage{biblatex}
\addbibresource{references.bib}

\usepackage[colorlinks=true,allcolors=blue!50]{hyperref}
\renewcommand{\sectionautorefname}{Sect.}
\renewcommand{\subsectionautorefname}{Sect.}
\renewcommand{\subsubsectionautorefname}{Sect.}
\Urlmuskip=0mu plus 1mu\relax

\lstdefinelanguage{Kotlin}{
  comment=[l]{//},
  morecomment=[s]{/*}{*/},
  emph={filter, first, firstOrNull, forEach, lazy, map, mapNotNull, println},
  keywords={!in, !is, abstract, actual, annotation, as, as?, break, by, catch, class, companion, const, constructor, continue, crossinline, data, delegate, do, dynamic, else, enum, expect, external, false, field, file, final, finally, for, fun, get, if, import, in, infix, init, inline, inner, interface, internal, is, lateinit, noinline, null, object, open, operator, out, override, package, param, private, property, protected, public, receiveris, reified, return, return@, sealed, set, setparam, super, suspend, tailrec, this, throw, true, try, typealias, typeof, val, var, vararg, when, where, while},
  ndkeywords={@Deprecated, @JvmField, @JvmName, @JvmOverloads, @JvmStatic, @JvmSynthetic, Array, Byte, Double, Float, Int, Integer, Iterable, Long, Runnable, Short, String, Any, Unit, Nothing},
  morestring=[b]",
  morestring=[s]{"""*}{*"""},
  sensitive=true,
}

\lstdefinelanguage{Java}{
  comment=[l]{//},
  morecomment=[s]{/*}{*/},
  emph={System, out, print, println, Arrays, Collections, List, Map, Set, stream},
  keywords={abstract, assert, boolean, break, byte, case, catch, char, class, const, continue, default, do, double, else, enum, extends, final, finally, float, for, if, goto, implements, import, instanceof, int, interface, long, native, new, null, package, private, protected, public, return, short, static, strictfp, super, switch, synchronized, this, throw, throws, transient, try, void, volatile, while, var},
  ndkeywords={@Deprecated, @FunctionalInterface, @Override, @SafeVarargs, @SuppressWarnings, @Documented, @Inherited, @Repeatable, @Retention, @Target, String, Integer, Double, Float, Long, Short, Boolean, Character, Object, Class, Runnable},
  morestring=[b]",
  morestring=[s]{"""*}{*"""},
  sensitive=true,
}

\lstset{
    backgroundcolor=\color{gray!10},
    basicstyle=\ttfamily\small,
    breaklines=true,
    breakatwhitespace=true,
    numbers=left,
    tabsize=2,
    captionpos=t,
    frame=single,
    rulecolor=\color{black},
    % Styles for Java and Kotlin
    numberstyle=\tiny\color{gray},
    keywordstyle=\color{blue}\bfseries,
    ndkeywordstyle=\color{BurntOrange}\bfseries,
    commentstyle=\color{green!50!black}\bfseries\itshape,
    stringstyle=\color{red},
    emphstyle=\color{Purple},
    identifierstyle=\color{black},
}

\title{\huge \bfseries Introduction to Kotlin}
\author{
  Christian Konersmann, Finn Paul Lippok, Paul Lukas \\[1ex]
  RWTH Aachen University, Germany \\
  \texttt{\{christian.konersmann,finn.lippok,paul.lukas\}@rwth-aachen.de}\\
  \and
  Proseminar: Advanced Programming Concepts \\
  Supervisor: Jan-Christoph Kassing
}
\date{\today}

\begin{document}
\maketitle

\begin{abstract}
  This paper is an introduction to Kotlin, a statically typed, object-oriented programming language designed to be fully interoperable with Java and the Java Virtual Machine (JVM).
  Kotlin offers a concise syntax, functional programming paradigms, and safety improvements compared to Java. In 2019, Google announced that Kotlin replaced Java as their preferred language for Android development.
\end{abstract}

\section{Introduction}
  Introduction, motivation, and goals of this paper.
  This paper assumes that the reader is familiar with the fundamentals of Java.
\section{Basic Syntax}
  This section will cover the basic syntax of Kotlin and highlight the changes compared to Java.
  The goal of this section is to provide a brief overview, focusing on the most important differences.

\subsection{Main Method}
  The main method is the entry point of every Java and Kotlin program.
  Java enforces object-oriented programming, thus requiring the main method to be declared inside a class.
  For the main method to be directly executable, the method must be declared as static and public.
  %If the main method were non-static, it would require an instance of the class to be created before it could be called, which in turn would require code execution before the main method could be called, thus creating a circular dependency.
  % ^ relevant? i would think so
  \begin{lstlisting}[language=Java,title={Java main method}]
public class Main {
  public static void main(String[] args) {
    System.out.println("Hello, World!");
  }
}
\end{lstlisting}
  Kotlin, on the other hand, does not require methods to be declared inside a class, allowing for a more functional programming style with top-level functions.
  These top-level functions can be called directly without the need to create an instance of a class, similar to static methods in Java\footnote{When compiling Kotlin to Java bytecode, top-level functions are compiled to static methods in a class named after the file name.} but without class affiliation.
  Kotlin further reduces boilerplate code by changing the default visibility of everything to public and allowing the main method to be declared without arguments passed as an array.
  Some further basic syntactical changes include making the semicolon optional and introducing the \textit{fun} keyword for defining functions.
  These changes lead to a more concise and readable main method and syntax in general.
\begin{lstlisting}[language=Kotlin,title={Kotlin main method}]
fun main() {
  println("Hello, World!")
}
\end{lstlisting}

\subsection{Type Declaration}
  In Kotlin, a variable declaration starts with the keyword \textit{val} for immutable variables or \textit{var} for mutable variables, similar to Java's \textit{final} and non-final variables.
  The type of a variable is declared after the variable name, separated by a colon.
\begin{lstlisting}[language=Java,title={Java data types}]
    final String name = "John Doe";
    int age = 42;
  \end{lstlisting}
  \begin{lstlisting}[language=Kotlin,title={Kotlin data types}]
    val name: String = "John Doe"
    var age: Int = 42
  \end{lstlisting}

\subsection{Type Inference}
  Kotlin also supports type inference, allowing the compiler to infer the type of a variable based on its initializer.
  \begin{lstlisting}[language=Kotlin]
    val name = "John Doe" // type is inferred as String
    var age = 42 // type is inferred as Int
  \end{lstlisting}

\subsection{Method Declaration}
  Similar to java. void = Unit which is optional.
  \begin{lstlisting}[language=Java,title={Java method declaration}]
    public int add(int a, int b) {
      return a + b;
    }
  \end{lstlisting}
  \begin{lstlisting}[language=Kotlin,title={Kotlin method declaration}]
    fun add(a: Int, b: Int): Int {
      return a + b
    }
  \end{lstlisting}

\subsection{Everything is an Object}
  In Kotlin, everything is an object, including primitive types and functions.

\section{New Language Constructs}
  This section focuses on the most important new language constructs that are not present in Java. 
  This section will illustrate Kotlin's advantages using a list of salespersons as an example and comparing it to Java.
  The example should represent a linked list of salespersons containing multiple attributes.
\subsection{Classes}
  In Java and Kotlin, classes are declared using the \textit{class} keyword. They can contain attributes, methods, and constructors.
  In this example, we will declare a class to hold information about a salesperson.
\begin{lstlisting}[language=Java,title={Java Class Declaration}]
public class SalesPerson {
  private final String name;
  private final int commissionRate;
  private double salesVolume;
  
  public SalesPerson(String name, int commissionRate, double transferAmount) {
      this.name = name;
      this.commissionRate = commissionRate;
      this.salesVolume = transferAmount;
    }
  }
\end{lstlisting}
  Kotlin improves upon Java by allowing the constructor to be declared directly within the class definition. As a reminder, public is the default visibility in Kotlin. In addition, classes are also final by default, meaning they cannot be inherited from unless explicitly declared as open.
  Furthermore, Kotlin allows for the declaration of attributes and their visibility directly within the constructor by adding the \textit{val} or \textit{var} keyword and the private keyword, resulting in a syntax very similar to Java records.
\begin{lstlisting}[language=Kotlin,title={Kotlin Class Declaration}]
  class SalesPerson(val name: String, private val commissionRate: Int, transferAmount: Double = 0.0) {
    var salesVolume: Double = transferAmount
  }
\end{lstlisting}
  In this example, both \textit{name} and \textit{commissionRate} are attributes, while \textit{transferAmount} is a normal constructor parameter. These constructor parameters can be used to initialize attributes of the class.
  It is still possible to declare attributes outside of the constructor, like \textit{salesVolume} in this example.
  In addition, Kotlin allows for default values for parameters in constructors and functions, which would require overloading in Java.
  
  %named parameters
\subsection{Properties}
\subsection{String Interpolation}

\subsection{Null Safety}
  Whenever a method or an attribute is called on a null reference in Java, a NullPointerException (NPE) is thrown. The concept behind Null Safety aims to reduce the occurrence of such NPEs. This is achieved through the advanced type system of Kotlin, which distinguishes between nullable and non-nullable types. This guarantees that variables of a non-nullable type can never be null. Unlike Java, this is enforced by the compiler at compile-time, therefore reducing possible sources of NPEs and enhancing the readability and robustness of the code. At runtime, both types are treated the same.
  
  By default, all types in Kotlin are non-nullable, meaning variables cannot hold a null value unless explicitly specified. To allow nullability, a question mark is appended to the type declaration.\footnote{This applies to both mutable and immutable variables.}

  \begin{lstlisting}[language=Kotlin]
    var a: String = "a is non-nullable"
    var b: String? = "b is nullable"
  \end{lstlisting}
  
\subsubsection{Null Safety Operators}
  When working with nullable types, you cannot directly access properties or methods because the value could be null, potentially causing an NPE. Whenever a nullable type is used, the value must be checked in some way to prevent the program from encountering an NPE. To avoid excessive use of if statements, Kotlin provides the safe call operator as a shortcut.

  The \textit{safe call operator} consists of the characters \texttt{?.} and is used when accessing a property or method of a nullable object. If the object is null, the operator returns null without evaluating the rest of the expression. Otherwise, the expression is evaluated as usual. Practically, this operator extends the already familiar dot notation for attributes and methods of objects. In principle, the safe call operator can also be seen as a shorthand for an if statement. By using the safe call operator, the code becomes much more readable and concise. With the reduced complexity, it is also less error-prone.

  \begin{lstlisting}[language=Kotlin,title={Using the safe call operator in comparison to an if statement}]
    var couldBeNull: String? = null
    println(if (couldBeNull == null) null else couldBeNull.length)
    println(couldBeNull?.length)    // Safe call operator
  \end{lstlisting}
  We can use multiple safe call operators and chain them together. The compiler evaluates the expression from left to right, checking each operator sequentially. If any value is null, the entire expression evaluates to null.
  Furthermore, the operator can also be used on the left side of an assignment. If the safe call operator evaluates to null, the assignment will be skipped. Otherwise, the value will be assigned as usual.
  % Wir sollten das Beispiel anpassen und die Formulierung in den zwei Absätzen drüber checken
  \begin{lstlisting}[language=Kotlin]
    var age: Int? = rwth?.ceo?.age // chained safe call operators
    rwth?.ceo?.age = 20 // assignment with chained operator
  \end{lstlisting}

  In Java, there is no safe call operator or anything comparable. To prevent NPE in Java, you have to explicitly check with an if-statement for the value to not be null. This is very inconvenient, hard to read, and prone to errors.
  \begin{lstlisting}[language=Java,title={Prevent NPE in Java}]
    String couldBeNull = null;
    if (couldBeNull == null) System.out.println("null");
    else System.out.println(couldBeNull.length());
  \end{lstlisting}
  \hfill \break

  The \textit{Elvis operator} is an extension of the safe call operator. The Elvis operator is mostly used in combination with the safe call operator. If the left side is null, the value will be set to a default value. Otherwise, the expression to the right-hand side will just be evaluated... TODO

\subsubsection{Smart Casts}
  Smart casts are another way to simplify the usage of nullable variables.

\section{Interoperability}
  In this chapter, the focus will be on interoperability between Java and Kotlin.
  In this context, interoperability means that the two languages are compatible with each other. Kotlin was designed to allow seamless integration of Java code in Kotlin and vice versa. This is possible because both languages compile to Java bytecode.This concept is especially useful when working with libraries. There are already countless libraries written in Java that can now be used in Kotlin, eliminating the need to rewrite a library with the same functionality specifically for Kotlin. This applies to both the official Java standard libraries and more specialized external libraries. Additionally, interoperability makes it much easier to migrate existing Java projects to Kotlin, as they do not need to be completely rewritten. This once again shows that Kotlin is a well-thought-out language designed to serve as an improvement over Java.

  \subsection{Call Java in Kotlin}
    One of the Key-Aspects is the interoperability with Java. This means we can use Java-written code inside of our Kotlin-Project.
    % Translate to english
    Kotlin wurde so designed, dass man in Kotlin jeglichen Java-code benutzen und ausführen kann. Nehmen wir an wir haben folgendes Beispiel einer Klasse, in der wir lediglich einfache Informationen speichern wollen.
    \begin{lstlisting}[language=Java,title={Example java class}]
      public class Salesman {
        private final String name;
        private int salary;

        public Salesman(String name, String title, int salary) {
          this.name = name;
          this.salary = salary;
        }

        public String getName() { return name; }

        public int getSalary() { return salary; }

        public void setSalary(int salary) { this.salary = salary; }
      }
    \end{lstlisting}
    Wenn wir in Kotlin jetzt auf diese Klasse zugreifen möchten und ein Objekt dieser erstellen wollen, so können wir mit der uns bekannten Kotlin-Syntax ein neues Objekt erstellen und darauf zugreifen. Da in Kotlin getter- und setter Methoden überflüssig sind, werden diese, sofern sie den java conventions bezüglich getter und setter methoden folgen, umgewandelt und können auch durch die Kotlin getter / setter erreicht werden. Wir nennen die dadurch entstehenden Attribute synthetic properties. Falls die Getter / Setter nicht nach den Java conventions geschrieben wurden, so lassen sich diese immernoch als normale Methoden aufrufen.
    \begin{lstlisting}[language=Kotlin, title={Access the Salesman class in Kotlin}]
      var carl = Salesman("carl mueller", 4500)
      println(carl.name) // prints 'carl mueller'
      carl.salary = 4600 // sets salary to 4600
      carl.setSalary(4600) // alternivly to the above
    \end{lstlisting}
    Kotlin erkennt, dass in der Java-Klasse das name-field final ist und nur ein Getter und kein Setter für dieses existiert. Dementsprechend wirft der compiler einen Fehler, wenn man dennoch versucht den Wert zu ändern. Die getter und setter verhalten sich also wie die aus Kotlin. Wenn nur ein Stter für das field existieren würde, so würde die Methode nicht zu einer synthetic propertie umgewandelt werden, da Kotlin keine set-only properties unterstützt. (Zitat?)
    
    There are a few keywords, such as \textit{in} or \textit{is}, that do not exist in Java; therefore, they are valid names for variables or similar identifiers. If there is Java code using those keywords, it is still possible to interact with this code through the backtick (`) character. In the following example, there is a method of a Java class called \textit{in} we want to access:
    \begin{lstlisting}[language=Kotlin]
      var salesman = Salesman("freddy", 1300)
      salesman.`in`(list)
    \end{lstlisting}

  \subsection{Call Kotlin in Java}
    It is possible to call Kotlin-Files in Java-code as well.

\subsection{Interoperability with Java}
  \begin{itemize}
    \item Explain the interoperability between Java and Kotlin (e.g. Using Java libraries in Kotlin)
    \item Use of annotations (e.g. \texttt{@JvmStatic}, \texttt{@JvmField}, \texttt{@JvmName}, \texttt{@JvmOverloads}) (out of scope for introduction?)
    \item Compile to other languages (e.g. JavaScript, Native) (out of scope for introduction?) => Ja, aber gehört unter anderen Punkt, vllt sowas wie Multiplatform development
  \end{itemize}

\section*{Acknowledgements}
  We would like to thank our instructor.

\end{document}
