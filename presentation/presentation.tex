\documentclass{beamer}
\usepackage[a4paper, left=3cm, right=3cm, top=3cm, bottom=3cm]{geometry}
\usepackage[dvipsnames]{xcolor}
\usepackage{listings}
\usepackage{hyperref}
\usepackage{titling}
\usepackage[T1]{fontenc}


\lstdefinelanguage{Kotlin}{
  comment=[l]{//},
  morecomment=[s]{/*}{*/},
  emph={filter, first, firstOrNull, forEach, lazy, map, mapNotNull, println},
  keywords={!in, !is, abstract, actual, annotation, as, as?, break, by, catch, class, companion, const, constructor, continue, crossinline, data, delegate, do, dynamic, else, enum, expect, external, false, field, file, final, finally, for, fun, get, if, import, in, infix, init, inline, inner, interface, internal, is, lateinit, noinline, null, object, open, operator, out, override, package, param, private, property, protected, public, receiveris, reified, return, return@, sealed, set, setparam, super, suspend, tailrec, this, throw, true, try, typealias, typeof, val, var, vararg, when, where, while},
  ndkeywords={@Deprecated, @JvmField, @JvmName, @JvmOverloads, @JvmStatic, @JvmSynthetic, Array, Byte, Double, Float, Int, Integer, Iterable, Long, Runnable, Short, String, Any, Unit, Nothing},
  morestring=[b]",
  morestring=[s]{"""*}{*"""},
  sensitive=true,
}

\lstdefinelanguage{Java}{
  comment=[l]{//},
  morecomment=[s]{/*}{*/},
  emph={System, out, print, println, Arrays, Collections, List, Map, Set, stream},
  keywords={abstract, assert, boolean, break, byte, case, catch, char, class, const, continue, default, do, double, else, enum, extends, final, finally, float, for, if, goto, implements, import, instanceof, int, interface, long, native, new, null, package, private, protected, public, return, short, static, strictfp, super, switch, synchronized, this, throw, throws, transient, try, void, volatile, while, var},
  ndkeywords={@Deprecated, @FunctionalInterface, @Override, @SafeVarargs, @SuppressWarnings, @Documented, @Inherited, @Repeatable, @Retention, @Target, String, Integer, Double, Float, Long, Short, Boolean, Character, Object, Class, Runnable},
  morestring=[b]",
  morestring=[s]{"""*}{*"""},
  sensitive=true,
}

\lstset{
    backgroundcolor=\color{gray!10},
    basicstyle=\ttfamily\small,
    breaklines=true,
    breakatwhitespace=true,
    numbers=left,
    tabsize=2,
    captionpos=t,
    frame=single,
    rulecolor=\color{black},
    % Styles for Java and Kotlin
    numberstyle=\tiny\color{gray},
    keywordstyle=\color{blue}\bfseries,
    ndkeywordstyle=\color{BurntOrange}\bfseries,
    commentstyle=\color{green!50!black}\bfseries\itshape,
    stringstyle=\color{red},
    emphstyle=\color{Purple},
    identifierstyle=\color{black},
}

\title{Kotlin}
\subtitle{Proseminar: Fortgeschrittene Programmierkonzepte}
\author[C. Konersmann, F. Lippok, P. Lukas]{
  Christian Konersmann, Finn Paul Lippok, Paul Lukas
}
\date{05.05.2025}
\colorlet{beamer@blendedblue}{kotlin-purple}

\begin{document}

\frame{\titlepage}

\begin{frame}{Was ist Kotlin?}
  \begin{columns}
    \begin{column}{0.7\textwidth}
      \begin{itemize}
        \item \textbf{Statisch typisierte} und \textbf{objektorientierte} Programmiersprache.
        \item \textbf{Basierend auf Java und der JVM} mit vollständiger \textbf{Interoperabilität} zu beiden.
      \end{itemize}
    \end{column}
    \begin{column}{0.3\textwidth}
      \begin{figure}
        \centering
        \includegraphics[width=0.6\textwidth]{Kotlin Full Color Logo Mark RGB.png}
      \end{figure}
    \end{column}
  \end{columns}
  \pause\vspace{0.5cm}
  \begin{itemize}
    \item \textbf{Wichtigste Vorteile gegenüber Java:} %Notes: Wichtigste Vorteile, auf die wir näher eingehen -> Erwähnen, wer welches Thema behandelt.
    \begin{itemize} %Sollten wir nochmal Interoperabilität als Vorteil erwähnen, da "leverage existing Java libraries"?
      \item Klare und präzise Syntax.
      \item Erweiterte Funktionen wie Null-Sicherheit.
      \item Umfassende Multiplattform-Entwicklungsmöglichkeiten.
    \end{itemize}
  \end{itemize}
\end{frame}

\begin{frame}[fragile]{Main-Methode}
  \begin{lstlisting}[language=Java, title=Java Main-Methode, xleftmargin=1em]
public class Main {
    public static void main(String[] args) {
        System.out.println("Hello, World!");
    }
}
  \end{lstlisting}
  \pause\vspace{0cm}
  \begin{lstlisting}[language=Kotlin, title=Kotlin Main-Methode, xleftmargin=1em]
fun main() {
    println("Hello, World!")
}
  \end{lstlisting}
    \pause\vspace{0cm}
    \begin{itemize}[<+->]
      \item Keine explizite Klassendeklaration erforderlich. %Notes: Methoden außerhalb einer Klasse sind quasi statisch, aber ohne Klassenzugehörigkeit. (Siehe auch println)
      \item Verwendung des Schlüsselworts \texttt{fun} zur Funktionsdeklaration.
      \item Standardzugriffsmodifikator ist \texttt{public}.
      \item \texttt{args}-Parameter ist optional.
      \item Semikolons sind nicht erforderlich.
    \end{itemize}
    \vspace{1cm}
\end{frame}

\begin{frame}[fragile]{Variablen-Deklaration}
  \begin{columns}
    \begin{column}{0.5\textwidth}
      \begin{lstlisting}[language=Java, title=Java, xleftmargin=1em]
int a = 5;
final String b = "Hallo";
      \end{lstlisting}
    \end{column}
    \begin{column}{0.5\textwidth}
      \begin{lstlisting}[language=Kotlin, title=Kotlin, xleftmargin=1em, numbers=none]
var a: Int = 5
val b: String = "Hallo"
      \end{lstlisting}
    \end{column}
  \end{columns}
  \vspace{0.5cm}
  \begin{itemize}[<+->]
    \item \texttt{var} für veränderliche Variablen, \texttt{val} für unveränderliche Variablen.
    \item Typangabe nach dem Variablennamen mit Doppelpunkt.
    \item In Kotlin gibt es \textbf{keine} primitiven Typen. %Notes: konsistenteres objektorientiertes Design. Auch Funktionen sind Objekte -> erlaubt funktionale Programmierung.
  \end{itemize}
  \pause\vspace{0.5cm}
  Kotlin unterstützt \textbf{Typinferenz}, d.h.\ der Typ kann weggelassen werden.
  \begin{itemize}
    \item Der Compiler leitet den Typ aus dem initialisierten Wert ab.
    \item Beispiel: \lstinline[language=kotlin]|var a = 5| ist auch möglich. % Dieses Beispiel weglassen und oben im Codeblock per Animation einfügen?
    %Notes Nutzt \textit{Constraint Solving}, ähnlich wie Unifikation in Haskell. 
  \end{itemize}
\end{frame}

\begin{frame}[fragile]{Klassen}
  \vspace{-0.25cm}
  \begin{lstlisting}[language=Java, title=Java, xleftmargin=1em]
public class Verkaufsperson {
  public final String name;
  private Double provision;

  public Verkaufsperson (String name, Double provision) {...}
}
  \end{lstlisting}
  \only<1>{\vspace{8.8\baselineskip}}
  \vspace{-0.25cm}
  \begin{onlyenv}<2>
    \begin{lstlisting}[language=Kotlin, title=Kotlin, xleftmargin=1em]
class Verkaufsperson() {



  val name: String
  private var provision: Double
}
    \end{lstlisting} 
  \end{onlyenv}
  \begin{onlyenv}<3>
    \begin{lstlisting}[language=Kotlin, title=Kotlin, xleftmargin=1em]
class Verkaufsperson(
  name: String,
  provision: Double = 0.2
) {
  val name: String = name
  private var provision: Double = provision
}
    \end{lstlisting} 
  \end{onlyenv}
  \begin{onlyenv}<4>
    \begin{lstlisting}[language=Kotlin, title=Kotlin, xleftmargin=1em]
class Verkaufsperson(
  val name: String,
  private var provision: Double = 0.2
) {


}
    \end{lstlisting} 
  \end{onlyenv}
  \begin{onlyenv}<5>
    \begin{lstlisting}[language=Kotlin, title=Kotlin, xleftmargin=1em]
class Verkaufsperson(
  val name: String,
  private var provision: Double = 0.2
) {}
    \end{lstlisting} 
    \begin{itemize}
      \item Ähnlich wie Java-Records, aber flexibler.
      \item Nur vererbbar, wenn als \texttt{open} deklariert.
    \end{itemize}
  \end{onlyenv}
\end{frame}

\begin{frame}[fragile]{Properties}
  \begin{onlyenv}<1>
    \begin{lstlisting}[language=Kotlin, title=Kotlin: Properties, xleftmargin=1em]
class Verkaufsperson(val name: String, 
    private var provision: Double = 0.2) {
            
  var umsatz : Int = 0



      
      
      
}
    \end{lstlisting}
  \end{onlyenv}
  \begin{onlyenv}<2>
    \begin{lstlisting}[language=Kotlin, title=Kotlin: Properties Zugriffsmodifikator, xleftmargin=1em]
class Verkaufsperson(val name: String, 
    private var provision: Double = 0.2) {
      
  var umsatz : Int = 0
    private set





}
    \end{lstlisting}
  \end{onlyenv}
  \begin{onlyenv}<3,4>
    \begin{lstlisting}[language=Kotlin, title=Kotlin: Benutzerdefinierte Zugriffsmethoden, xleftmargin=1em]
class Verkaufsperson(val name: String, 
    private var provision: Double = 0.2) {

  var umsatz : Int = 0
    private set(value) {
      if (value < 0)
        throw IllegalArgumentException("Umsatz muss positiv sein")
      field = value
    }
}
    \end{lstlisting}
  \end{onlyenv}
  \begin{uncoverenv}<4>
    \begin{itemize}
    \item Punkt-Notation ruft automatisch Setter/Getter auf.
    \item Beispiel: \lstinline[language=kotlin]|verkaufsperson.umsatz = -1| wirft eine \texttt{IllegalArgumentException}.
    % Notes: 
    \end{itemize}
  \end{uncoverenv}
\end{frame}

%TODO
  %Methoden return type
  %story? -> Unser Ziel ist es eine Verkaufsperson Klasse zu erstellen um ...


% Null safety

\begin{frame}[fragile]{Null Safety}
  Motiviation: Null safety
  \begin{lstlisting}[language=Kotlin, title=Java example, xleftmargin=1em]
Verkaufsperson person = null;
System.out.println(person.name);
  \end{lstlisting}
  \begin{itemize}
    \item Exception in thread "main" java.lang.NullPointerException %Falsche Anführungszeichen?
    \item Kann zu Programmabbruch führen oder weitere Fehler nach sich ziehen
  \end{itemize}
\end{frame}

\begin{frame}[fragile]{Null Safety}
  \begin{itemize}
    \item unterscheidung zwischen nullable types und non-nullable types
    \item Programmierer muss Null safety gewährleisten
  \end{itemize}
  \begin{lstlisting}[language=Kotlin]
var a : String = "a ist non-nullable"
var b : String? = "b ist nullable"
  \end{lstlisting}
\end{frame}

\begin{frame}[fragile]{Null Safety: Safe Call Operator}
  Ziel: sicherer Zugriff auf Datenfeld
  \begin{lstlisting}[language=Java, title=in Java]
private final SalesPerson supervisor;

public void printSupervisor() {
  if (supervisor == null)
    System.out.println("null");
  else System.out.println(supervisor.name);
}
  \end{lstlisting}
  \begin{lstlisting}[language=Kotlin, title=in Kotlin]
val supervisor: SalesPerson? = null

fun printSupervisor() {
  println(supervisor?.name)
} 
  \end{lstlisting}
\end{frame}

\begin{frame}[fragile]{Null Safety: Safe call Operator}
  \begin{lstlisting}[language=Kotlin]
val name: String? = supervisor?.supervisor?.name   
  \end{lstlisting}
  \begin{lstlisting}[language=Kotlin]
supervisor?.supervisor?.salesVolume = 0.0
  \end{lstlisting}
\end{frame}

\begin{frame}[fragile]{Null Safety: Elvis Operator}
  \begin{lstlisting}[language=Java]
public void printSupervisor() {
  if (supervisor == null)
    System.out.println("No supervisor");
  else System.out.println(supervisor.name);
}   
  \end{lstlisting}
  \begin{lstlisting}[language=Kotlin]
fun printSupervisor() {
  println(supervisor?.name ?: "No supervisor")
}
  \end{lstlisting}
\end{frame}

\begin{frame}[fragile]{Null Safety: Elvis Operator}
  \begin{lstlisting}[language=Java]
public void printSupervisor() {
  if (supervisor == null)
    System.out.println("No supervisor");
  else System.out.println(supervisor.name);
}
  \end{lstlisting}
  \begin{lstlisting}[language=Kotlin]
fun printSupervisor() {
  println(supervisor?.name ?: "No supervisor")
}
  \end{lstlisting}
\end{frame}

\begin{frame}[fragile]{Null Safety: Not-null assertion}
  \begin{lstlisting}[language=Java]
val couldBeNull: String? = null
var b: String = possiblyNull!!
  \end{lstlisting}
  \begin{itemize}
    \item Kann zu NullPointerExceptions führen
  \end{itemize}
\end{frame}

\begin{frame}[fragile]{Null Safety: Nullable Receiver}
  \begin{itemize}
    \item Baut auf Extension functions
    \item erlaubt Methodenaufruf auf nullable types
    \item Null Werte werden innerhalb der Methode behandelt
  \end{itemize}
  \begin{lstlisting}[language=Kotlin]
fun SalesPerson?.print() {
  if (this == null) return println("This person does not exist.")
  return println("$name: $salesVolume sold")
}
  \end{lstlisting}
  \begin{lstlisting}[language=Kotlin]
var sales: SalesPerson? = null
sales.print() // This person does not exist
  \end{lstlisting}
\end{frame}

\begin{frame}[fragile]{Interoperabilität}
  \begin{lstlisting}[language=Java]
public class SalesPerson {
  private final String name;
  private double salesVolume;
  
  public SalesPerson(String name, Double salesVolume) {
    this.name = name;
    this.salesVolume = salesVolume;
  }

  public String getName() { return name; }
  public Double getSalesVolume() { return salesVolume; }
  public void setSalesVolume(Double salesVolume) { this.salesVolume = salesVolume; }
}
  \end{lstlisting}
\end{frame}

\begin{frame}[fragile]{Interoperability}
  \begin{lstlisting}[language=Java]
var carl = SalesPerson("Carl Mueller", 4500.0)
println(carl.name)
carl.salesVolume = 4600.0
carl.setSalesVolume(4600.0)
  \end{lstlisting}
  \begin{itemize}
    \item Kotlin erstellt synthetic properties
    \item Aufruf über getter/setter Methoden weiterhin möglich
  \end{itemize}
\end{frame}

\begin{frame}[fragile]{Interoperability: Mapped types}
  \begin{itemize}
    \item \texttt{java.lang.Object} $\Rightarrow$ \texttt{kotlin.Any!}
    \item \texttt{java.lang.Integer} $\Rightarrow$ \texttt{kotlin.Int?}
    \item Primitiver Typ \texttt{int} $\Rightarrow$ \texttt{kotlin.Int}
    \item Rückgabewert \texttt{void} $\Rightarrow$ \texttt{Unit}
  \end{itemize}
\end{frame}

\begin{frame}[fragile]{Interoperability: Null safety mit Java}
  \begin{itemize}
    \item Aus Java zurückgegebene Instanzen können \texttt{null} sein
    \item haben spezial-Typ: platform type
    \item gelockerte Regeln bezüglich Null safety
    \item anfälliger für NullPointerExceptions
  \end{itemize}
  \begin{lstlisting}[language=Java]
public SalesPerson createSalesPerson() {
  return null;
}
  \end{lstlisting}
  \begin{lstlisting}[language=Kotlin]
val sales: SalesPerson = createSalesPerson()
println(sales.name)
  \end{lstlisting}
\end{frame}

\begin{frame}[fragile]{Interoperability: Java Arrays in Kotlin}
  \begin{itemize}
    \item Es gibt keine primitiven Typen in Kotlin
    \item Extra Kotlin Klassen für primitive Arrays
  \end{itemize}
  \begin{lstlisting}[language=Java]
public static void takeArray(int[] array) { ... }
  \end{lstlisting}
  \begin{lstlisting}[language=Kotlin]
var array: IntArray = intArrayOf(1, 2, 3)
takeArray(array)
  \end{lstlisting}
\end{frame}

\begin{frame}[fragile]{Interoperability: Kotlin Properties in Java}
  \begin{lstlisting}[language=Kotlin]
var name: String
    \end{lstlisting}
    \begin{lstlisting}[language=Java]
private String name;
public String getName() { return name; }
public void setName(String name) { this.name = name; }
      \end{lstlisting}
\end{frame}

\begin{frame}[fragile]{Interoperability: Instance Fields}
  \begin{lstlisting}[language=Java]
SalesPerson person = new SalesPerson("carl", 0.0);
System.out.println(person.name);
  \end{lstlisting}
  \begin{itemize}
    \item Aufruf von Attributen durch Punkt-Notation
    \item In Java ist der Aufruf durch die \texttt{@JvmField} Annotation möglich
  \end{itemize}
\end{frame}

\begin{frame}[fragile]{Interoperability: Instance Fields}
  \begin{lstlisting}[language=Kotlin]
class SalesPerson (@JvmField var name:String) {}
  \end{lstlisting}
  \begin{lstlisting}[language=Java]
public void example() {
  SalesPerson person = new SalesPerson("Carl", 0.0);
  System.out.println(person.name);
}
  \end{lstlisting}
\end{frame}

\end{document}